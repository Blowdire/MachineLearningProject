%%%%%%%%%%%%%%%%%%%%%%%%%%%%%%%%%%%%%%%%%
% Wenneker Article
% LaTeX Template
% Version 2.0 (28/2/17)
%
% This template was downloaded from:
% http://www.LaTeXTemplates.com
%
% Authors:
% Vel (vel@LaTeXTemplates.com)
% Frits Wenneker
%
% License:
% CC BY-NC-SA 3.0 (http://creativecommons.org/licenses/by-nc-sa/3.0/)
%
%%%%%%%%%%%%%%%%%%%%%%%%%%%%%%%%%%%%%%%%%

%----------------------------------------------------------------------------------------
%	PACKAGES AND OTHER DOCUMENT CONFIGURATIONS
%----------------------------------------------------------------------------------------

\documentclass[10pt, a4paper, twocolumn]{article} % 10pt font size (11 and 12 also possible), A4 paper (letterpaper for US letter) and two column layout (remove for one column)
\usepackage{multirow}
%%%%%%%%%%%%%%%%%%%%%%%%%%%%%%%%%%%%%%%%%
% Wenneker Article
% Structure Specification File
% Version 1.0 (28/2/17)
%
% This file originates from:
% http://www.LaTeXTemplates.com
%
% Authors:
% Frits Wenneker
% Vel (vel@LaTeXTemplates.com)
%
% License:
% CC BY-NC-SA 3.0 (http://creativecommons.org/licenses/by-nc-sa/3.0/)
%
%%%%%%%%%%%%%%%%%%%%%%%%%%%%%%%%%%%%%%%%%

%----------------------------------------------------------------------------------------
%	PACKAGES AND OTHER DOCUMENT CONFIGURATIONS
%----------------------------------------------------------------------------------------

\usepackage[english]{babel} % English language hyphenation

\usepackage{microtype} % Better typography

\usepackage{amsmath,amsfonts,amsthm} % Math packages for equations

\usepackage[svgnames]{xcolor} % Enabling colors by their 'svgnames'

\usepackage[hang, small, labelfont=bf, up, textfont=it]{caption} % Custom captions under/above tables and figures

\usepackage{booktabs} % Horizontal rules in tables

\usepackage{lastpage} % Used to determine the number of pages in the document (for "Page X of Total")

\usepackage{graphicx} % Required for adding images

\usepackage{enumitem} % Required for customising lists
\setlist{noitemsep} % Remove spacing between bullet/numbered list elements

\usepackage{sectsty} % Enables custom section titles
\allsectionsfont{\usefont{OT1}{phv}{b}{n}} % Change the font of all section commands (Helvetica)

%----------------------------------------------------------------------------------------
%	MARGINS AND SPACING
%----------------------------------------------------------------------------------------

\usepackage{geometry} % Required for adjusting page dimensions

\geometry{
	top=1cm, % Top margin
	bottom=1.5cm, % Bottom margin
	left=2cm, % Left margin
	right=2cm, % Right margin
	includehead, % Include space for a header
	includefoot, % Include space for a footer
	%showframe, % Uncomment to show how the type block is set on the page
}

\setlength{\columnsep}{7mm} % Column separation width

%----------------------------------------------------------------------------------------
%	FONTS
%----------------------------------------------------------------------------------------

\usepackage[T1]{fontenc} % Output font encoding for international characters
\usepackage[utf8]{inputenc} % Required for inputting international characters

\usepackage{XCharter} % Use the XCharter font

%----------------------------------------------------------------------------------------
%	HEADERS AND FOOTERS
%----------------------------------------------------------------------------------------

\usepackage{fancyhdr} % Needed to define custom headers/footers
\pagestyle{fancy} % Enables the custom headers/footers

\renewcommand{\headrulewidth}{0.0pt} % No header rule
\renewcommand{\footrulewidth}{0.4pt} % Thin footer rule

\renewcommand{\sectionmark}[1]{\markboth{#1}{}} % Removes the section number from the header when \leftmark is used

%\nouppercase\leftmark % Add this to one of the lines below if you want a section title in the header/footer

% Headers
\lhead{} % Left header
\chead{\textit{\thetitle}} % Center header - currently printing the article title
\rhead{} % Right header

% Footers
\lfoot{} % Left footer
\cfoot{} % Center footer
\rfoot{\footnotesize Page \thepage\ of \pageref{LastPage}} % Right footer, "Page 1 of 2"

\fancypagestyle{firstpage}{ % Page style for the first page with the title
	\fancyhf{}
	\renewcommand{\footrulewidth}{0pt} % Suppress footer rule
}

%----------------------------------------------------------------------------------------
%	TITLE SECTION
%----------------------------------------------------------------------------------------

\newcommand{\authorstyle}[1]{{\large\usefont{OT1}{phv}{b}{n}\color{DarkRed}#1}} % Authors style (Helvetica)

\newcommand{\institution}[1]{{\footnotesize\usefont{OT1}{phv}{m}{sl}\color{Black}#1}} % Institutions style (Helvetica)

\usepackage{titling} % Allows custom title configuration

\newcommand{\HorRule}{\color{DarkGoldenrod}\rule{\linewidth}{1pt}} % Defines the gold horizontal rule around the title

\pretitle{
	\vspace{-30pt} % Move the entire title section up
	\HorRule\vspace{10pt} % Horizontal rule before the title
	\fontsize{32}{36}\usefont{OT1}{phv}{b}{n}\selectfont % Helvetica
	\color{DarkRed} % Text colour for the title and author(s)
}

\posttitle{\par\vskip 15pt} % Whitespace under the title

\preauthor{} % Anything that will appear before \author is printed

\postauthor{ % Anything that will appear after \author is printed
	\vspace{10pt} % Space before the rule
	\par\HorRule % Horizontal rule after the title
	\vspace{20pt} % Space after the title section
}

%----------------------------------------------------------------------------------------
%	ABSTRACT
%----------------------------------------------------------------------------------------

\usepackage{lettrine} % Package to accentuate the first letter of the text (lettrine)
\usepackage{fix-cm}	% Fixes the height of the lettrine

\newcommand{\initial}[1]{ % Defines the command and style for the lettrine
	\lettrine[lines=3,findent=4pt,nindent=0pt]{% Lettrine takes up 3 lines, the text to the right of it is indented 4pt and further indenting of lines 2+ is stopped
		\color{DarkGoldenrod}% Lettrine colour
		{#1}% The letter
	}{}%
}

\usepackage{xstring} % Required for string manipulation

\newcommand{\lettrineabstract}[1]{
	\StrLeft{#1}{1}[\firstletter] % Capture the first letter of the abstract for the lettrine
	\initial{\firstletter}\textbf{\StrGobbleLeft{#1}{1}} % Print the abstract with the first letter as a lettrine and the rest in bold
}

%----------------------------------------------------------------------------------------
%	BIBLIOGRAPHY
%----------------------------------------------------------------------------------------

\usepackage[backend=bibtex,style=authoryear,natbib=true]{biblatex} % Use the bibtex backend with the authoryear citation style (which resembles APA)

\addbibresource{example.bib} % The filename of the bibliography

\usepackage[autostyle=true]{csquotes} % Required to generate language-dependent quotes in the bibliography
 % Specifies the document structure and loads requires packages

%----------------------------------------------------------------------------------------
%	ARTICLE INFORMATION
%----------------------------------------------------------------------------------------

\title{Machine learning} % The article title

\author{
	Davide Abete, Fabrizio Cominetti e Agazzi Ruben % Authors
}

% Example of a one line author/institution relationship
%\author{\newauthor{John Marston} \newinstitution{Universidad Nacional Autónoma de México, Mexico City, Mexico}}

\date{\today} % Add a date here if you would like one to appear underneath the title block, use \today for the current date, leave empty for no date

%----------------------------------------------------------------------------------------

\begin{document}

\maketitle % Print the title

\thispagestyle{firstpage} % Apply the page style for the first page (no headers and footers)

%----------------------------------------------------------------------------------------
%	ABSTRACT
%----------------------------------------------------------------------------------------
\tableofcontents
\lettrineabstract{Il progetto consiste nell'analisi di vari modelli di regressione applicati ad un dataset contenente le informazioni ambientali riferite alla temperatura globale a partire dal Gennaio 1850 al Novembre 2015.}


%----------------------------------------------------------------------------------------
%	ARTICLE CONTENTS
%----------------------------------------------------------------------------------------

\section{Introduzione}
Il cambiamento climatico è la più grande minaccia del nostro tempo.
Nell'ultimo secolo la temperatura globale è sempre più aumentata.%citare magari una ricerca
Lo scopo del progetto consiste nel realizzare e analizzare vari modelli di regressione per prevedere la temperatura media globale del mese successivo, sulla base dei dati rilevati durante la mensilità precedente.La scelta di questo progetto è stata influenzata dalle tematiche riguardanti il cambiamento climatico, un tema in costante ascesa di interesse e di importanza sempre più rilevante.
Per effettuare questa previsione abbiamo usato un dataset contenente varie misure di temperatura media, rilevate ogni mese a partire dal Gennaio 1750 al Novembre 2015.


%------------------------------------------------

\section{Dataset}

Il Dataset utilizzato è "Climate Change: Earth Surface Temperature Data".%citazione
Il Dataset è stato ottenuto da Kaggle, che è una versione ripulita del Dataset formito dalla "Berkley Earth", un'organizzazione non-profit indipendente specializatta in "Environmental Data Science".

Fra i vari Dataset forniti su Kaggle è stato scelto quello riguardante le temperature globali.

\subsection{Descrizione del Dataset}
Il Dataset è composto da 3192 righe e da 9 colonne.
Le colonne sono le seguenti:
\begin{itemize}
	\item dt: Data di rilevamento dei dati
	\item landAverageTemperature: temperatura media globale del terreno espressa in gradi Celsius.
	\item LandAverageTemperatureUncertainity: Intervallo di confidenza al $95 \%$ intorno alla media della temperatura globale
	\item LandMaxTemperature: media della temperatura massima globale del terreno espressa in gradi Celsius
	\item LandMaxTemperatureUncertainty:  Intervallo di confidenza al $95 \%$ intorno alla media della temperatura massima globale
	\item LandMinTemperature: media della temperatura minima globale del terreno espressa in gradi Celsius
	\item LandMinTemperatureUncertainty: Intervallo di confidenza al $95 \%$ intorno alla media della temperatura minima globale
	\item LandAndOceanAverageTemperature: Temperatura media globale del terrestre e oceanica espressa in gradi Celsius.
	\item LandAndOceanAverageTemperatureUncertainty: Intervallo di confidenza al $95 \%$ intorno alla media della temperatura media terrestre e oceanica globale.
\end{itemize}
\subsection{Esplorazione dei dati}
Per la fase di esplorazione dei dati è stata utilizzata la piattaforma Knime, in particolare è stato usato il nodo Statistics, attraverso il quale sono stati rilevati i seguenti indici e numero di missing values.
%\begin{tabular}{| m{5em} | m{1cm}| m{1cm} | m{1cm} |}
%\hline Colonna & Minimo & Massimo & Media & Deviazione Standard
%
%\end{tabular}
\begin{center}
\begin{tabular}{||c c c c||} 
 \hline
 Colonna  & Media & STD & Missing V. \\ [0.5ex] 
 \hline\hline
	landAvgTemp &   8.3747 & 4.3813 & 12 \\
\hline\hline
	LandAvgTempUnc &   0.9385& 1.0964 & 12 \\
	\hline\hline
	LandMaxTemp & 14.3506 & 4.3096 & 1200 \\
	\hline\hline
	LandMaxTempUnc & 0.4798 & 0.5832 & 1200 \\
	\hline\hline
	LandMinTemp &2.7436 & 4.1558 & 1200 \\
	\hline\hline
	LandMinTempUnc & 0.4318 & 0.4458 & 1200 \\
	\hline\hline
	LndOcnAvgTemp & 15.2126 & 1.2741 & 1200 \\
	\hline\hline
	LndOcnAvgTempUnc & 0.1285 & 0.0736 & 1200 \\[0.5ex] \hline 
\end{tabular}
\end{center}

La fase di esplorazione ha permesso di identificare la variabile target che ci interessa prevedere, ovvero \textit{landAverageTemperature} del mese successivo, e le colonne restanti che fungono da attributi esplicativi.
La variabile target selezionata è di tipo numerico, continua e assume valori fra $[-2.080, 19.021]$ con 3 cifre decimali.

\section{Pre-Processing}
Durante la fase di data exploration si è potuto osservare che quasi tutte le colonne hanno 1200 valori mancanti, in quanto le osservazioni effettuate prima del Gennaio 1850 non sono stati rilevati tali dati. 
\subsection{Missing Values}
Per risolvere questa problematica si è deciso di rimuovere le righe che presentano missing values, eliminando quindi 1200 righe e riducendo l'intervallo di tempo dei valori che ora partono dal Gennaio 1850 invece che dal Gennaio 1750. Per eliminare le righe che presentano missing values è stato utilizzato il nodo di Knime chiamato \textit{<<Missing Values>>}

\subsection{Data Augmentation}
Per poter allenare i vari classificatori ci serve il dato relativo alla temperatura media terrestre del mese successivo. Per fare questo abbiamo ordinato decrescente i dati in base alla colonna della data utilizzando il nodo \textit{<<Sorter>>} e in seguito abbiamo utilizzato il nodo \textit{<<Lag Column>>} per creare una nuova colonna contenente la temperatura media del mese successivo.

\subsection{Selezione delle Variabili}
Per selezionare le variabili da utilizzare per l'apprendimento dei classificatori è stato utilizzato un filtro di correlazione in modo da eliminare attributi ridondanti. Il valore soglia di correlazione scelto è di 0.9. Alla fine del processo si feature selection sono state tenute 6 colonne su 10. Per effettuare questa operazione sono stati utilizzati i nodi \textit{<<Linear Correlation>>} e \textit{<<Correlation Filter>>}
\section{Modelli di Regressione}
%----------------------------------------------------------------------------------------
%	BIBLIOGRAPHY
%----------------------------------------------------------------------------------------

\printbibliography[title={Bibliography}] % Print the bibliography, section title in curly brackets

%----------------------------------------------------------------------------------------

\end{document}
