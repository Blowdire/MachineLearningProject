%%%%%%%%%%%%%%%%%%%%%%%%%%%%%%%%%%%%%%%%%
% Wenneker Article
% LaTeX Template
% Version 2.0 (28/2/17)
%
% This template was downloaded from:
% http://www.LaTeXTemplates.com
%
% Authors:
% Vel (vel@LaTeXTemplates.com)
% Frits Wenneker
%
% License:
% CC BY-NC-SA 3.0 (http://creativecommons.org/licenses/by-nc-sa/3.0/)
%
%%%%%%%%%%%%%%%%%%%%%%%%%%%%%%%%%%%%%%%%%

%----------------------------------------------------------------------------------------
%	PACKAGES AND OTHER DOCUMENT CONFIGURATIONS
%----------------------------------------------------------------------------------------

\documentclass[10pt, a4paper, twocolumn]{article} % 10pt font size (11 and 12 also possible), A4 paper (letterpaper for US letter) and two column layout (remove for one column)
\usepackage{multirow}
\input{structure.tex} % Specifies the document structure and loads requires packages

%----------------------------------------------------------------------------------------
%	ARTICLE INFORMATION
%----------------------------------------------------------------------------------------

\title{Machine learning} % The article title

\author{
	Davide Abete, Fabrizio Cominetti e Agazzi Ruben % Authors
}

% Example of a one line author/institution relationship
%\author{\newauthor{John Marston} \newinstitution{Universidad Nacional Autónoma de México, Mexico City, Mexico}}

\date{\today} % Add a date here if you would like one to appear underneath the title block, use \today for the current date, leave empty for no date

%----------------------------------------------------------------------------------------

\begin{document}

\maketitle % Print the title

\thispagestyle{firstpage} % Apply the page style for the first page (no headers and footers)

%----------------------------------------------------------------------------------------
%	ABSTRACT
%----------------------------------------------------------------------------------------
\tableofcontents
\lettrineabstract{Il progetto consiste nell'analisi di vari modelli di regressione applicati ad un dataset contenente le informazioni ambientali riferite alla temperatura globale a partire dal Gennaio 1850 al Novembre 2015.}


%----------------------------------------------------------------------------------------
%	ARTICLE CONTENTS
%----------------------------------------------------------------------------------------

\section{Introduzione}
Il cambiamento climatico è da molti definito come 'la più grande minaccia del nostro tempo'.
L'ultimo secolo si è reso protagonista di un aumento graduale e preoccupante della temperatura globale, ed oggi questo tema ha assunto rilevanza cruciale in numerosi dibattiti politici relativi al futuro del pianeta.%citare magari una ricerca
Scopo di questo progetto è di realizzare modelli di machine learning con il fine di prevedere le future temperature medie globali. Precisamente, saranno utilizzati modelli di regressione per prevedere la temperatura del mese successivo a quello preso in considerazione.
(Lo scopo di questo lavoro consiste nel realizzare e successivamente analizzare vari modelli di regressione con il fine di prevedere la temperatura media globale del mese successivo, sulla base dei dati rilevati durante la mensilità precedente.
Come specificato in apertura di sezione, la scelta di questo progetto è stata influenzata dal crescente interesse verso le tematiche riguardanti il cambiamento climatico, sia da un punto di vista personale che, appunto, globale.
I dati utilizzati per realizzare questo progetto sono scaricabili dalla piattaforma Kaggle al seguente indirizzo () e contengono diverse misure di temperature medie, rilevate ogni mese a partire dal Gennaio 1750 al Novembre 2015.


%------------------------------------------------

\section{Dataset}

Il dataset utilizzato è intitolato "Climate Change: Earth Surface Temperature Data".%citazione
Come precisato poco sopra, il dataset è stato ottenuto tramite la piattaforma Kaggle, sulla quale è stata resa disponibile una versione ripulita del dataset fornito dalla "Berkeley Earth", un'organizzazione non-profit indipendente specializatta in temi di 'environmental data science'.
L'organizzazione Berkeley Earth definisce il problema del global warming come 'la sfida definitiva del nostro tempo' e pone l'accento sulla necessità di ottenere con urgenza dati di qualità e informazioni scientifiche di valore sul tema.
Il progetto è stato svolto principalmente tramite l'utilizzo della piattaforma Knime e, in particolari casi, del linguaggio di programmazione Python.

\subsection{Descrizione del dataset}
Il dataset è composto da 3192 record e dai seguenti 9 attributi:
\begin{itemize}
	\item dt: Data di rilevamento dei dati
	\item landAverageTemperature: temperatura media globale del terreno espressa in gradi Celsius.
	\item LandAverageTemperatureUncertainity: Intervallo di confidenza al $95 \%$ intorno alla media della temperatura globale
	\item LandMaxTemperature: media della temperatura massima globale del terreno espressa in gradi Celsius
	\item LandMaxTemperatureUncertainty:  Intervallo di confidenza al $95 \%$ intorno alla media della temperatura massima globale
	\item LandMinTemperature: media della temperatura minima globale del terreno espressa in gradi Celsius
	\item LandMinTemperatureUncertainty: Intervallo di confidenza al $95 \%$ intorno alla media della temperatura minima globale
	\item LandAndOceanAverageTemperature: Temperatura media globale del terrestre e oceanica espressa in gradi Celsius.
	\item LandAndOceanAverageTemperatureUncertainty: Intervallo di confidenza al $95 \%$ intorno alla media della temperatura media terrestre e oceanica globale.
\end{itemize}
\subsection{Esplorazione dei dati}
Per la fase di esplorazione dei dati è stato utilizzato principalmente il nodo Statistics, presente all'interno della piattaforma Knime, attraverso il quale sono stati rilevati alcuni indici di posizione, variabilità e il numero di missing values, dei quali ne elencheremo i principali di seguito:
%\begin{tabular}{| m{5em} | m{1cm}| m{1cm} | m{1cm} |}
%\hline Colonna & Minimo & Massimo & Media & Deviazione Standard
%
%\end{tabular}
\begin{center}
\begin{tabular}{||c c c c||} 
 \hline
 Colonna  & Media & STD & Missing V. \\ [0.5ex] 
 \hline\hline
	landAvgTemp &   8.3747 & 4.3813 & 12 \\
\hline\hline
	LandAvgTempUnc &   0.9385& 1.0964 & 12 \\
	\hline\hline
	LandMaxTemp & 14.3506 & 4.3096 & 1200 \\
	\hline\hline
	LandMaxTempUnc & 0.4798 & 0.5832 & 1200 \\
	\hline\hline
	LandMinTemp &2.7436 & 4.1558 & 1200 \\
	\hline\hline
	LandMinTempUnc & 0.4318 & 0.4458 & 1200 \\
	\hline\hline
	LndOcnAvgTemp & 15.2126 & 1.2741 & 1200 \\
	\hline\hline
	LndOcnAvgTempUnc & 0.1285 & 0.0736 & 1200 \\[0.5ex] \hline 
\end{tabular}
\end{center}

La fase di esplorazione ha permesso di identificare la variabile target che ci interessa prevedere, ovvero \textit{landAverageTemperature} del mese successivo, e le colonne restanti che fungono da attributi esplicativi.
La variabile target selezionata è di tipo numerico, continua e assume valori fra $[-2.080, 19.021]$ con 3 cifre decimali.

\section{Pre-processing}
Durante la fase di data exploration si è potuto osservare che molte colonne presentavano un numero di valori mancanti pari a 1200. Ciò è dovuto al fatto che per le osservazioni effettuate prima del Gennaio 1850 non sono stati rilevati tali dati di interesse. 
\subsection{Missing values}
Per risolvere questa problematica si è deciso di rimuovere i record che presentano missing values, eliminando quindi 1200 righe e riducendo l'intervallo di tempo dei valori che, a questo punto, partono dal Gennaio 1850 invece che dal Gennaio 1750. Per eliminare le righe che presentano missing values è stato utilizzato il nodo di Knime chiamato \textit{<<Missing Values>>}

\subsection{Data augmentation}
Per essere in grado di allenare i vari classificatori è necessario il dato relativo alla temperatura media terrestre del mese successivo. Per fare questo abbiamo ordinato in modo decrescente i dati in base alla colonna contenente la data utilizzando il nodo \textit{<<Sorter>>} e in seguito abbiamo utilizzato il nodo \textit{<<Lag Column>>} per creare una nuova colonna contenente la temperatura media del mese successivo.

\subsection{Selezione delle variabili}
Per selezionare le variabili da utilizzare per l'apprendimento dei classificatori è stato utilizzato un filtro di correlazione in modo da eliminare attributi ridondanti. Il valore soglia di correlazione scelto è di 0.9. Alla fine del processo di feature selection sono state tenute 6 colonne su 10. Per effettuare questa operazione sono stati utilizzati i nodi \textit{<<Linear Correlation>>} e \textit{<<Correlation Filter>>}
\section{Modelli}
%----------------------------------------------------------------------------------------
%	BIBLIOGRAPHY
%----------------------------------------------------------------------------------------

\printbibliography[title={Bibliography}] % Print the bibliography, section title in curly brackets

%----------------------------------------------------------------------------------------

\end{document}
